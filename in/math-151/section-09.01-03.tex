::: Section 9.1--9.3

Write the nth-term formula for the following sequences
\begin{enumerate}
\item \(\{3,7,11,15,\dots\}\)
\item \(\{2,-1,\frac{1}{2},-\frac{1}{4},\dots\}\)
\item \(\{1,x,\frac{x^2}{2},\frac{x^3}{6},\frac{x^4}{24},\frac{x^5}{120},\dots\}\)
\end{enumerate}
===
\begin{enumerate}
\item\(a_n=4n-1\)
\item\(a_n={(-1)}^{n+1}2^{2-n}\)
\item\(a_n=\frac{x^{n-1}}{(n-1)!}\)
\end{enumerate}

---

Find the limit of the sequence and state whether the sequence converges or
diverges
\[a_n=\frac{n}{n+1}\]
===
\(\lim=1\), therefore the sequence converges.

---

Find the limit of the sequence and state whether the sequence converges or
diverges
\[a_n=2+{(-1)}^n\]
===
The function oscillates between the values 1 and 3, thus limit does not
exist, and therefore the sequence diverges.

---

Find the limit of the sequence and state whether the sequence converges or
diverges
\[a_n={\left(1+\frac{1}{n}\right)}^n\]
===
\(\lim=e\), therefore the sequence converges.

---

Find the limit of the sequence and state whether the sequence converges or
diverges
\[\left\{1,\sfrac{4}{3},\sfrac{9}{7},\sfrac{16}{15},\sfrac{25}{31}\right\}\]
===
\(\lim=0\), therefore the sequence converges.

---

A sequence is \textit{monotonic} if all of its terms are entirely either:
\begin{enumerate}
  \item \rule{1cm}{0.15mm}
    (\(a_1\,\rule{1cm}{0.15mm}\,\cdots\,\rule{1cm}{0.15mm}\,a_n\)), or
  \item \rule{1cm}{0.15mm}
    (\(a_1\,\rule{1cm}{0.15mm}\,\cdots\,\rule{1cm}{0.15mm}\,a_n\)).
\end{enumerate}
===
\begin{enumerate}
  \item \underline{non-decreasing}
    (\(a_1\,\underline{\leq}\,\cdots\,\underline{\leq}\,a_n\)), or
  \item \underline{non-increasing}
    (\(a_1\,\underline{\geq}\,\cdots\,\underline{\geq}\,a_n\)).
\end{enumerate}

---

A sequence is bounded \rule{1cm}{0.15mm} if there is a number \(M\) such that
\(a_n\leq M\) for all \(n\).
===
\underline{above}

---

A sequence is bounded \rule{1cm}{0.15mm} if there is a number \(M\) such that
\(a_n\geq M\) for all \(n\).
===
\underline{below}

---

A sequence is bounded if it is\\
\rule{1\linewidth}{0.15mm}.
===
\underline{bounded both above and below}

---

A geometric series is a series of the form
\[\sum_{n=0}^\infty\underline{\text{\hspace{0.5cm}?\hspace{0.5cm}}}\]
===
\[\sum_{n=0}^\infty ar^n\]

---

A geometric series with ratio \(r\) will:
\begin{enumerate}
  \item Diverge if \rule{1cm}{0.15mm}.
  \item Converge to \(S=\rule{1cm}{0.15mm}\) if \rule{1cm}{0.15mm}.
\end{enumerate}
===
\begin{enumerate}
  \item Diverge if \(|r|\geq 1\).
  \item Converge to \(\displaystyle S=\frac{a}{1-r}\) if
    \(0<|r|<1\).
\end{enumerate}

---

Does the following series converge or diverge? If it converges, find the sum.
\[\sum_{n=0}^\infty \frac{1}{2^n}\]
===
Converges, and the sum equals 2.

---

Does the following series converge or diverge? If it converges, find the sum.
\[\sum_{n=0}^\infty e^{-n}\]
===
Converges, and the sum equals \(\frac{e}{e-1}\).

---

Does the following series converge or diverge? If it converges, find the sum.
\[\sum_{n=0}^\infty {\left(\frac{3}{2}\right)}^2\]
===
Diverges

---

Consider the repeating decimal \(0.1313\cdots\); Convert this decimal to a
fraction.
===
\[0.1313\cdots=\frac{13}{99}\]
